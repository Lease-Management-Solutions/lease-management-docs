% --- Document: Preamble ---
\documentclass[../main.tex]{subfiles}
\graphicspath{{\subfix{../assets/}}}
% --- Document: Start ---
\begin{document}
% Closing Remarks
\section{APÊNDICE}
\subsection{Manual de Utilização do Sistema}

% Login
\subsubsection{Login}
Ao acessar o sistema, o usuário deverá informar seu e-mail e senha para autenticação. Uma
vez autenticado, o usuário terá acesso ao sistema, e será armezenado um token de sessão,
que será utilizado para manter o usuário autenticado. Quando o usuário sair do sistema, o
token de sessão será invalidado. O login será assegurado por meio de criptografia de senha.
% Login Screen
\subsubsection{Tela de Login}
Na tela inicial de login, tem um campo para o usuário informar seu e-mail, e um campo para 
o usuário informar sua senha, um botão para o servirá para o usuário clicar e acessar o 
sistema. Um link com a descrição \textit{"Não tem conta? Cadastre-se"}, abaixo do botão 
de login será para o usuário clicar e ser redirecionado para a tela de cadastro. A caixa 
de seleção \textit{"Ocultar/Exibir Senha"} será para o usuário marcar ou desmarcar, 
conforme desejar, para ocultar ou exibir a senha digitada. E por ultimo, um link com a
descrição \textit{"Esqueceu a senha?"}, que será para o usuário clicar e ser redirecionado
para a tela de recuperação de senha, que utilizará condigo de recuperação de 5 digitos.
% New User
\subsubsection{Novo Usuário}
Para se cadastrar no sistema, o usuário deverá informar seu nome, data de nascimento, 
cidade de moradia, gênero sexual, estado civil, e-mail, senha e confirmar a senha. 
O e-mail deverá ser único, e o sistema deverá verificar se o e-mail informado é válido. 
O usuário deverá confirmar a senha digitada, e o sistema deverá verificar se as senhas
digitadas são iguais. O usuário deverá aceitar os termos de uso do sistema para se 
cadastrar. Uma vez cadastrado, o usuário poderá acessar o sistema.
% Main Feed
\subsubsection{Feed Principal}
O feed principal será a tela inicial do usuário autenticado, onde ele poderá visualizar
os posts de outros usuários, e poderá postar suas próprias publicações. O feed principal
será atualizado automaticamente a cada nova publicação de outros usuários ou dele próprio. 
O usuário poderá curtir, comentar e compartilhar as publicações de outros usuários. Na 
parte superior do feed principal, haverá um campo de pesquisa, onde o usuário poderá
pesquisar por outros usuários, ou por publicações específicas. O feed principal será
dividido em páginas, e o usuário poderá navegar entre as páginas para visualizar publicações
mais antigas. Abaixo do campo de pesquisar, será o campo para o usuário postar suas
próprias publicações. O usuário poderá postar textos, imagens, vídeos e links. O usuário
poderá marcar outros usuários em suas publicações, e poderá adicionar hashtags. O usuário
poderá escolher se sua publicação será pública, ou se será visível apenas para seus amigos.
% Sidebar
\subsubsection{Menu Lateral}
O menu lateral será um menu que ficará na lateral esquerda da tela, e será acessível em
qualquer tela do sistema. O menu lateral deverá conter atalhos para as principais 
funcionalidades, como o feed principal, o perfil do usuário, a lista de amigos, as
configurações do sistema, e a opção de sair do sistema. O menu lateral deverá conter
também a foto do usuário, seu nome, e seu nível de amizade com o usuário autenticado.
% Publication on the Feed
\subsubsection{Publicação no Feed}
Visualmente todas as publicações serão exibidas de forma semelhante, com o mesmo 
tamanho visual, com um cabeçalho contendo a foto do usuário que fez a publicação,
seu nome, a data e hora da publicação, e um menu de opções. O menu de opções
deverá conter as opções de curtir, comentar, compartilhar, e denunciar a publicação. 
A publicação será composta de texto, imagens, vídeos, links, hashtags, e marcações de
usuários. Se houver muito texto, deverá ser exibido um botão de "Leia Mais", que
ao ser clicado, exibirá o texto completo. Se houver muitas imagens, deverá ser exibido
um botão de "Ver Mais", que ao ser clicado, exibirá as imagens restantes. Se houver
muitos comentários, deverá ser exibido um botão de "Ver Mais", que ao ser clicado,
exibirá os comentários restantes.
% User Feed
\subsubsection{Feed do usuário}
O feed do usuário será uma tela onde o usuário poderá visualizar suas próprias publicações,
ou as publicações de outros usuários que ele tenha acessado. O feed do usuário deverá 
apresentar a classificação pessoal dos usuários, que será baseada nas notas que os usuários
deram para suas publicações, conforme as interações que tiveram com elas e com os imóveis, 
para que seja categorizado com um bom locador, ou um bom locatário, etc.
% Publications
\subsubsection{Publicações}
A publicações terão suas próprias classificações, uma vez que o intuito do sistema é
fazer com que os usuários interajam com as publicações, e com os imóveis. Os imóveis
publicados, poderão ter as classificações de "Bom", "Ruim", "Regular", "Ótimo", "Péssimo".
As publicações serão categorizadas em "Venda" ou "Aluguel" (quando o usuário esta 
vendendo ou alugando o seu imóvel), "Compro" ou "Alugo" (quando o usuário procura um 
imóvel para comprar ou alugar).
% Publication Detail
\subsubsection{Detalhamento das Publicações}
Além das classificação, do autor do post, e todos os metadados envolvidos com a elaboração 
do post, o post deverá conter uma padronização de informações, como: "Título", "Descrição",
"Localização", "Preço", "Área", "Quartos", "Banheiros", "Vagas de Garagem", "Mobiliado", 
etc..., seguindo todas as regras de negocios e detalhamento do imóvel.
% --- Document: End ---
\end{document}

